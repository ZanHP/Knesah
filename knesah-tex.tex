\documentclass[a4paper,12pt]{article}

\usepackage[slovene]{babel}
\usepackage{amsfonts,amssymb,amsmath}
\usepackage[utf8]{inputenc}
\usepackage[T1]{fontenc}
\usepackage{lmodern}
\usepackage{graphicx}

\newtheorem{izrek}{Izrek}
\newtheorem{trditev}{Trditev}
\newtheorem{posledica}{Posledica}
\newtheorem{lema}{Lema}
\newtheorem{pripomba}{Pripomba}
\newtheorem{definicija}{Definicija}
\newtheorem{zgled}{Zgled}

\title{Kromatično število Kneserjevih grafov \\ 
\Large Seminar}
\author{Žan Hafner Petrovski \\
Fakulteta za matematiko in fiziko \\
Oddelek za matematiko}
\date{12.\ maj 2017}

\begin{document}

%%%%
 
\maketitle

%%%%

\section{Uvod}

V teoriji grafov poznamo mnogo različnih tipov grafov. Razlikujemo jih glede na njihove specifične lastnosti. V tem seminarju se bomo ukvarjali s {\em Kneserjevimi grafi} oziroma podrobneje, s {\em kromatičnim številom} le-teh.

%%%%

\begin{definicija}
Graf $K(n,k)$, $n \geq k \geq 1$ in $n, k \in \mathbb{N}$, imenujemo Kneserjev, če je množica vozlišč $V(n,k)$ družina vseh podmnožic moči $k$ množice $\{1, 2, \ldots, n\}$. Dve vozlišči pa sta povezani natanko takrat, ko sta disjunktni.
\end{definicija}

\begin{thebibliography}{1}

\bibitem{AiZ}
M.~Aigner in G.~M.~Ziegler, \emph{Proofs from THE BOOK}, 2.\ izdaja, Springer, Berlin--Heidelberg--New York, 2001.
\bibitem{Ber}
%https://math.berkeley.edu/{tu manjka ~}brandtm/research/kneser.pdf (25.4.2017)

\end{thebibliography}

\end{document}