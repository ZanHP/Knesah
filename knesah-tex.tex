\documentclass[a4paper,12pt]{article}

\usepackage[slovene]{babel}
\usepackage{amsfonts,amssymb,amsmath}
\usepackage[utf8]{inputenc}
\usepackage[T1]{fontenc}
\usepackage{lmodern}
\usepackage{graphicx}

\newtheorem{izrek}{Izrek}
\newtheorem{trditev}{Trditev}
\newtheorem{posledica}{Posledica}
\newtheorem{lema}{Lema}
\newtheorem{opomba}{Opomba}
\newtheorem{definicija}{Definicija}
\newtheorem{zgled}{Zgled}

\title{Kromatično število Kneserjevih grafov \\ 
\Large Seminar}
\author{Žan Hafner Petrovski \\
Fakulteta za matematiko in fiziko \\
Oddelek za matematiko}
\date{12.\ maj 2017}

\begin{document}

%%%%
 
\maketitle

%%%%

\section{Uvod}

V teoriji grafov poznamo mnogo različnih tipov grafov. Razlikujemo jih glede na njihove specifične lastnosti. V tem seminarju se bomo ukvarjali s {\em Kneserjevimi grafi} oziroma podrobneje, s {\em kromatičnim številom} le-teh. Najprej podajmo nekaj glavnih definicij.

%%%%

\begin{definicija}
Graf $K(n,k)$, $n \geq k \geq 1$ in $n, k \in \mathbb{N}$, imenujemo \mbox{\textbf{Kneserjev}}, če je množica vozlišč $V(n,k)$ družina vseh $k$-elementnih podmnožic množice $\{1, 2, \ldots, n\}$. Dve vozlišči sta povezani natanko takrat, ko sta disjunktni. 
\end{definicija}

Povejmo še, da za število vozlišč velja $|V(n,k)|={{n}\choose{k}}$. V primeru, ko je $n < 2k$, imata vsaki dve $k$-elementni množici neprazen presek. Tak Kneserjev graf nima nobenih povezav, zato privzemimo, da velja $n \geq 2k$.


\begin{definicija}
Najmanjše število $m$, ki zadošča barvanju vozlišč grafa $G$, imenujemo \textbf {Kromatično število}. Označimo ga s $\chi(K(n,k)).$
\end{definicija}

\begin{definicija}
Preslikavo $c: V \rightarrow \{1, \ldots, m\}$, ki slika vozlišča grafa v množico barv, imenujemo \textbf {barvanje}. Ta preslikava zadošča pogoju, da sta vsaki dve sosednji vozlišči pobarvani z različnima barvama.
\end{definicija}

Kromatično število grafa $G$ je torej najmanjše število barv, s katerimi lahko pobarvamo vozlišča grafa tako, da se nobeni dve sosednji vozlišči slikata v isto barvo. Množico vozlišč $V$ bi radi predstavili kot disjunktno unijo barvnih razredov $V = V_1 \sqcup V_2 \sqcup \ldots \sqcup V_{\chi(G)}$, teh pa želimo, da je najmanj. ??Za vsak barvni razred velja, da imajo vsi njegovi elementi, torej $k$-elementne množice, neprazen presek.??

Vozlišča Kneserjevega grafa $K(n,k)$ bomo razdelili na disjunktne množice $V = V_1 \sqcup V_2 \sqcup \ldots \sqcup V_{\chi(K(n,k))}$, kjer bo vsak $V_i$ družina množic moči $k$ z nepraznim presekom. Ker smo predpostavili, da je $n \geq 2k$, poenostavimo zapis in pišimo $n = 2k + d, k \geq 1, d \geq 0$.

\section{Kneserjeva domneva}

Cilj tega seminarja je dokazati naslednji izrek: \\

\begin{izrek}[Kneser]
Za kromatično število Kneserjevega grafa velja
$$\chi(K(2k+d,k)) = d+2.$$
\end{izrek}

\noindent
Preformulirajmo ta izrek v obliko problema obstoja na sledeč način: \\

\noindent
{\em Če družino podmnožic s $k$ elementi množice $\{1, 2, \ldots, 2k+d\}$ razdelimo na $d+1$ razredov,  $V = V_1 \sqcup V_2 \sqcup \ldots \sqcup V_{d+1}$, potem obstaja $i$, da $V_i$ vsebuje par $k$-elementnih disjunktnih množic $A$ in $B$.} \\

Tako smo prišli do splošnejše različice izreka. Ta nam, še preden se poglobimo v sam dokaz, nudi drugačen pogled na zastavljen problem, saj ne omenja grafov. László Lovász je uvidel, da bistvo problema tiči v slavnem izreku o $d$-dimenzionalni enotski sferi $S^d$ v $\mathbb{R}^d $, $S^d = \{x \in \mathbb{R}: |x|=1\}$. Zapišimo še ta izrek.

\begin{izrek}[Borsuk-Ulam]
Za vsako zvezno preslikavo $f:S^d \rightarrow \mathbb{R}^d$, z $d$-sfere v $d$-prostor, obstajata antipodni točki $x^*$ in $-x^*$, ki ju $f$ slika v isto točko, torej $f(x^*)=f(-x^*)$.
\end{izrek}

Dokaz tega izreka lahko bralec najde v knjigi ''Using the Borsuk-Ulam theorem'' matematika Jirija Matouška, mi pa se bomo posvetili njegovi uporabi pri dokazu izreka Lyusternika in Shnirel'mana.

\begin{izrek}[Lyusternik-Shnirel'man]
Če je $d$-sfera $S^d$ pokrita z $d+1$ množicami,
$$S^d = U_1 \cup U_2 \cup \ldots \cup U_d \cup U_{d+1},$$
tako, da so vse izmed prvih $d$ množic $U_1, U_2, \ldots, U_d$ bodisi odprte bodisi zaprte, potem ena izmed $d+1$ množic vsebuje par antipodnih točk $x^*$ in $-x^*$.
\end{izrek}

\noindent
{\em Dokaz (s protislovjem in uporabo Borsuk-Ulamovega izreka):} Naj bo pokritje $S^d = U_1 \cup U_2 \cup \ldots \cup U_d \cup U_{d+1}$ dano, kot je zapisano v izreku. Predpostavimo, da noben izmed $U_i$ ne vsebuje dveh antipodnih točk. Definirajmo preslikavo $f:S^d \rightarrow \mathbb{R}^d$ na sledeč način:
$$f(x) := (d(x,U_1), d(x,U_2), \ldots, d(x,U_d)).$$
Tu $d(x,U_i)$ označuje razdaljo med točko $x$ in množico $U_i$. Ker je to zvezna funkcija na $x$, je tudi $f$ zvezna. Torej lahko uporabimo Borsuk-Ulamov izrek, ki nam pove, da na domeni $f$, torej na $S^d$, obstajata antipodni točki $x^*$ in $-x^*$ z lastnostjo $f(x^*)=f(-x^*)$. Ker po predpostavki $U_{d+1}$ ne vsebuje antipodnih točk, sklepamo, da je vsaj en izmed $x^*$ in $-x^*$ vsebovan v eni izmed množic $U_i$, recimo v $U_k$ za $k\leq d$. Brez škode za splošnost lahko privzamemo, da je to $x^*$, torej $x^* \in U_k$. To pomeni, da je $d(x^*, U_k) = 0$, ključno pa je, da je zaradi lastnosti $f(x^*)=f(-x^*)$ tudi $d(-x^*, U_k) = 0$.\\
Obravnavajmo najprej primer, ko je $U_k$ zaprt. Potem iz $d(-x^*, U_k) = 0$ sledi, da je $-x^* \in U_k$, kar pa je protislovje s predpostavko, da noben izmed $U_i$ ne vsebuje para antipodnih točk.\\
Če je $U_k$ odprt, potem iz $d(-x^*, U_k) = 0$ sledi, da $-x^*$ leži v zaprtju $U_k$, torej v $\overline {U_k}$. Ta množica pa je %ne razumem nekih simbolov....

%%%%

\begin{thebibliography}{1}

\bibitem{AiZ}
M.~Aigner in G.~M.~Ziegler, \emph{Proofs from THE BOOK}, 2.\ izdaja, Springer, Berlin--Heidelberg--New York, 2001.
\bibitem{Ber}
%https://math.berkeley.edu/{tu manjka ~}brandtm/research/kneser.pdf (25.4.2017)

\end{thebibliography}

\end{document}