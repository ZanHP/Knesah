\documentclass[a4paper,12pt]{article}

\usepackage[slovene]{babel}
\usepackage{amsfonts,amssymb,amsmath}
\usepackage[utf8]{inputenc}
\usepackage[T1]{fontenc}
\usepackage{lmodern}
\usepackage{graphicx}

\newtheorem{izrek}{Izrek}
\newtheorem{trditev}{Trditev}
\newtheorem{posledica}{Posledica}
\newtheorem{lema}{Lema}
\newtheorem{opomba}{Opomba}
\newtheorem{definicija}{Definicija}
\newtheorem{zgled}{Zgled}

\title{Kromatično število Kneserjevih grafov \\ 
\Large Seminar}
\author{Žan Hafner Petrovski \\
Fakulteta za matematiko in fiziko \\
Oddelek za matematiko}
\date{12.\ maj 2017}

\begin{document}

%%%%
 
\maketitle

%%%%

\section{Uvod}

V teoriji grafov poznamo mnogo različnih tipov grafov. Razlikujemo jih glede na njihove specifične lastnosti. V tem seminarju se bomo ukvarjali s {\em Kneserjevimi grafi} oziroma podrobneje, s {\em kromatičnim številom} le-teh. Najprej podajmo nekaj glavnih definicij.

%%%%

\begin{definicija}
Graf $K(n,k)$, $n \geq k \geq 1$ in $n, k \in \mathbb{N}$, imenujemo \mbox{\textbf{Kneserjev}}, če je množica vozlišč $V(n,k)$ družina vseh podmnožic moči $k$ množice $\{1, 2, \ldots, n\}$. Dve vozlišči sta povezani natanko takrat, ko sta disjunktni. 
\end{definicija}

Povejmo še, da za število vozlišč velja $|V(n,k)|={{n}\choose{k}}$. V primeru, ko je $n < 2k$, imata vsaki dve množici s $k$ elementi neprazen presek. Tak Kneserjev graf nima nobenih povezav, zato privzemimo, da velja $n \geq 2k$.


\begin{definicija}
Najmanjše število $m$, ki zadošča barvanju vozlišč grafa $G$, imenujemo \textbf {Kromatično število}. Označimo ga s $\chi(K(n,k)).$
\end{definicija}

\begin{definicija}
Preslikavo $c: V \rightarrow \{1, \ldots, m\}$, ki slika vozlišča grafa v množico barv, imenujemo \textbf {barvanje}. Ta preslikava zadošča pogoju, da sta vsaki dve sosednji vozlišči pobarvani z različnima barvama.
\end{definicija}

Kromatično število grafa $G$ je torej najmanjše število barv, s katerimi lahko pobarvamo vozlišča grafa tako, da se nobeni dve sosednji vozlišči slikata v isto barvo. Množico vozlišč $V$ bi radi predstavili kot disjunktno unijo barvnih razredov $V = V_1 \sqcup V_2 \sqcup \ldots \sqcup V_{\chi(G)}$, teh pa želimo, da je najmanj. ??Za vsak barvni razred velja, da imajo vsi njegovi elementi, torej množice moči $k$, neprazen presek.??

Vozlišča Kneserjevega grafa $K(n,k)$ bomo razdelili na disjunktne množice $V = V_1 \sqcup V_2 \sqcup \ldots \sqcup V_{\chi(K(n,k))}$, kjer bo vsak $V_i$ družina množic moči $k$ z nepraznim presekom. Ker smo predpostavili, da je $n \geq 2k$, poenostavimo zapis in pišimo $n = 2k + d, k \geq 1, d \geq 0$.

\section{Kneserjeva domneva}
\begin{izrek}%[Kneserjeva domneva]
Za kromatično število Kneserjevega grafa velja
$$\chi(K(2k+d,k)) = d+2.$$
\end{izrek}

\begin{thebibliography}{1}

\bibitem{AiZ}
M.~Aigner in G.~M.~Ziegler, \emph{Proofs from THE BOOK}, 2.\ izdaja, Springer, Berlin--Heidelberg--New York, 2001.
\bibitem{Ber}
%https://math.berkeley.edu/{tu manjka ~}brandtm/research/kneser.pdf (25.4.2017)

\end{thebibliography}

\end{document}