\documentclass{beamer}

\usepackage[slovene]{babel}
\usepackage{amsfonts,amssymb}
\usepackage[utf8]{inputenc}
\usepackage{lmodern}
\usepackage[T1]{fontenc}

\usetheme{Warsaw}

\def\qed{$\hfill\Box$}   % konec dokaza
\newtheorem{izrek}{Izrek}
\newtheorem{trditev}{Trditev}
\newtheorem{posledica}{Posledica}
\newtheorem{lema}{Lema}
\newtheorem{definicija}{Definicija}
\newtheorem{pripomba}{Pripomba}
\newtheorem{primer}{Primer}
\newtheorem{zgled}{Zgled}
\newtheorem{zgledi}{Zgledi uporabe}
\newtheorem{zglediaf}{Zgledi aritmetičnih funkcij}
\newtheorem{oznaka}{Oznaka}

\title{Kromatično število Kneserjevih grafov}
\author{Žan Hafner Petrovski}
\institute{Fakulteta za matematiko in fiziko \\
Oddelek za matematiko}
\date{12.\ maj 2017}

\begin{document}

%%%%

\begin{frame}
\titlepage
\end{frame}

%%%%

\begin{frame}{Definicije}

\begin{definicija}
Graf $K(n,k)$, $n \geq k \geq 1$ in $n, k \in \mathbb{N}$, imenujemo \mbox{\alert{Kneserjev}}, če je množica vozlišč $V(n,k)$ družina vseh $k$-elementnih podmnožic množice $\{1, 2, \ldots, n\}$. Dve vozlišči sta povezani natanko takrat, ko sta disjunktni.
\end{definicija}

\pause

\begin{definicija}
Preslikavo $c: V \rightarrow \{1, \ldots, m\}$, ki slika vozlišča grafa v množico barv, imenujemo \alert {barvanje vozlišč grafa}. Barvanje vozlišč je pravilno, če sta vsaki dve sosednji vozlišči pobarvani z različnima barvama.
\end{definicija}

\pause

\begin{definicija}
Najmanjše naravno število $m$, za katero obstaja pravilno barvanje vozlišč grafa $G$ z $m$ barvami, imenujemo \textbf {kromatično število}. Označimo ga s $\chi(G).$
\end{definicija}

\end{frame}

%%%%

\begin{frame}{Zgled}

{\em Petersenov graf} oziroma $K(5,2)$.

\begin{figure}[h!]
	\centering
	\begin{minipage}{0.45\textwidth}
		\centering
		\includegraphics[width=0.8\textwidth]{petersenov_graf_barvanje} % first figure itself
        	\caption{Primer barvanja s $3$ barvami}
    	\end{minipage}\hfill
    	\begin{minipage}{0.45\textwidth}
       	 \centering
        	 \includegraphics[width=0.8\textwidth]{petersenov_graf_mnozice} % second figure itself
       	 \caption{Prikaz povezav med disjunktnimi množicami}
    	\end{minipage}
\end{figure}

\end{frame}

%%%%

\begin{frame}{Kneserjeva domneva}

\begin{trditev}
Vozlišča Kneserjevega grafa $K(2k+d,k)$ lahko pobarvamo z $d+2$ barvama.
\end{trditev}

\pause

\begin{izrek}[Kneser]
Za kromatično število Kneserjevega grafa velja
$$\chi(K(2k+d,k)) = d+2.$$
\end{izrek}

\pause

\begin{izrek}[Ekvivalentno]
Če družino $k$-elementnih podmnožic množice $\{1, 2, \ldots, 2k+d\}$ razdelimo na $d+1$ razredov,  $V = V_1 \sqcup V_2 \sqcup \ldots \sqcup V_{d+1}$, potem obstaja $i$ tako, da $V_i$ vsebuje par disjunktnih $k$-elementnih množic $A$ in $B$.
\end{izrek}

\end{frame}

%%%%

\begin{frame}{Potrebovali bomo:}

\begin{izrek}[Borsuk-Ulam]
Za vsako zvezno preslikavo $f:S^d \rightarrow \mathbb{R}^d$ $d$-sfere v $d$-prostor, obstajata antipodni točki $x^*$ in $-x^*$, ki ju $f$ slika v isto točko, torej $f(x^*)=f(-x^*)$.
\end{izrek}

\pause

\begin{izrek}[Lyusternik-Shnirel'man]
Če je $d$-sfera $S^d$ pokrita z $d+1$ množicami,
$$S^d = U_1 \cup U_2 \cup \ldots \cup U_d \cup U_{d+1},$$
tako, da je vsaka izmed prvih $d$ množic $U_1, U_2, \ldots, U_d$ bodisi odprta bodisi zaprta, potem ena izmed $d+1$ množic vsebuje par antipodnih točk $x^*$ in $-x^*$.
\end{izrek}

\end{frame}

%%%%

\begin{frame}{Splošna lega točk na sferi}

\begin{definicija}
Točke iz množice $\{1,2,\ldots,2k+d\}$ so v \textbf {splošni legi} na sferi $S^{d+1} \subset \mathbb{R}^{d+2}$, če nobenih $d+2$ točk iz omenjene množice ne leži na hiperravnini skozi središče sfere.
\end{definicija}

\begin{figure}[h!]
\centering
\includegraphics[width=0.35\textwidth]{splosna_lega}
\caption{Primer za $d=0$, postavitev $4$ točk sfero $S^1$}
\end{figure}

\end{frame}

\end{document}